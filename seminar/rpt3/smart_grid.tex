\documentclass[9pt]{article}
\usepackage{graphicx}
\usepackage[margin=0.5in]{geometry}
\setlength{\parindent}{0cm}

\usepackage{listings}
\usepackage{color}

\definecolor{dkgreen}{rgb}{0,0.6,0}
\definecolor{gray}{rgb}{0.5,0.5,0.5}
\definecolor{mauve}{rgb}{0.58,0,0.82}

\lstset{frame=tb,
  language=Python,
  aboveskip=3mm,
  belowskip=3mm,
  showstringspaces=false,
  columns=flexible,
  basicstyle={\small\ttfamily},
  numbers=none,
  numberstyle=\tiny\color{gray},
  keywordstyle=\color{blue},
  commentstyle=\color{dkgreen},
  stringstyle=\color{mauve},
  breaklines=true,
  breakatwhitespace=true
  tabsize=3
}

\begin{document}

\title{Managing Demand Response and Renewables in Smart Grid}
\author{Kyle Kastner}

\maketitle

\subsection*{Presenter}
Dr. Nikolaos Gatsis 
\subsection*{What is the research trying to do?}
Dr. Gatsis' research is focused on smart grid scheduling using convex optimization techniques.
\subsection*{Articulate the objectives using absolutely no jargon}
Scheduling of smart appliances is a critical objective for the "smart grid", and is crucial for future growth.
\subsection*{How is it done today, and what are the limits of current practice?}
Currently, no scheduling at all is performed (at least at the grid level), which results in 
huge amounts of inefficiency and waste for both consumers and producers of electrical power.
\subsection*{What's new in the approach and why do we think it will be successful?}
This scheduling is currently done using convex optimization techniques, built from special models
for different types of devices.While convex methods may not fully cover all the problems, they are
a very good start for stepping into the future of electrical distribution.
\subsection*{Who cares?}
More efficient electrical distribution affects every single person who uses electricity, as reducing electrical waste
can save money for every customer.
\subsection*{If successful, what difference will it make?}
We will be able to use electrical energy more efficiently, reducing fossil fuel consumption, saving 
costs in both infrastructure and bills. It also increases the effectiveness of green energy sources.
\subsection*{What are the risks and the payoffs?}
This research is fairly low risk, as there is little existing literature. The payoff may be enormous,
if a highly efficient scheduling algorithm is created, tested, and adopted in industry.
\subsection*{How much will it cost?}
Simulated testing of this research will cost nearly nothing, as the improvements are algorithmic.
\subsection*{Is it economically feasible?}
It will be many years before a solution is implemented, and current limitations for the smart grid are 
lack of infrastructure, complex politics, and consumer buy-in. However, this research is important
in making sure that good technical solutions are available when these other issues are solved.
\end{document}
