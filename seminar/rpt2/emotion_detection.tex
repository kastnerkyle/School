\documentclass[9pt]{article}
\usepackage{graphicx}
\usepackage[margin=0.5in]{geometry}
\setlength{\parindent}{0cm}

\usepackage{listings}
\usepackage{color}

\definecolor{dkgreen}{rgb}{0,0.6,0}
\definecolor{gray}{rgb}{0.5,0.5,0.5}
\definecolor{mauve}{rgb}{0.58,0,0.82}

\lstset{frame=tb,
  language=Python,
  aboveskip=3mm,
  belowskip=3mm,
  showstringspaces=false,
  columns=flexible,
  basicstyle={\small\ttfamily},
  numbers=none,
  numberstyle=\tiny\color{gray},
  keywordstyle=\color{blue},
  commentstyle=\color{dkgreen},
  stringstyle=\color{mauve},
  breaklines=true,
  breakatwhitespace=true
  tabsize=3
}

\begin{document}

\title{Improving the Robustness of Emotional Speech Detection Systems}
\author{Kyle Kastner}

\maketitle

%\section*{Anomaly Detection on Vehicle Networks}
%\subsection*{Kyle Kastner}
\subsection*{Presenter}
Dr. Carlos Busso, University of Texas at Dallas 
\subsection*{What is the research trying to do?}
This research is attempting to increase the robustness and accuracy of systems for analyzing emotions in human speech.
\subsection*{Articulate the objectives using absolutely no jargon}
The objectives of this research are to find better mathematical models for detecting different emotions in human speech.
\subsection*{How is it done today, and what are the limits of current practice?}
Emotion detection in speech is currently performed similarly to many other machine learning methods for speech - building features,
applying classifiers, and tweaking the methods based on classification error.
\subsection*{What's new in the approach and why do we think it will be successful?}
This approach uses modern methods in machine learning, especially in the modeling stage. These newer models
provide addtional accuracy and more importantly, stability in the decision process.
\subsection*{Who cares?}
Emotion detection is one of the most difficult problems in speech and natural language processing. Inflection
of wording is often MORE important than the words themselves... ever company which does speech recognition could
gain accuracy from this work.
\subsection*{If successful, what difference will it make?}
Speech recognition systems will recognize more than the words themselves, but the intent behind them. Calming angry customers, 
matching joyous orders - many businesses would be interested in this.
\subsection*{What are the risks and the payoffs?}
The risks are fairly minimal, as this work is incremental advancement in a field. 
\subsection*{How much will it cost?}
Research in machine learning methods ususally has minimal cost, as it is mostly mathematical or statistical advances which 
create better systems.
\subsection*{Is it economically feasible?}
This project is very feasible, due to the low cost, and the great progress available in the case of success
\end{document}
