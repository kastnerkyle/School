\documentclass[conference]{IEEEtran}

\begin{document}
%
% paper title
% can use linebreaks \\ within to get better formatting as desired
\title{Memory Architectures and Caching Techniques for GPU Computing}

% author names and affiliations
% use a multiple column layout for up to three different
% affiliations
\author{\IEEEauthorblockN{Kyle Kastner}
\IEEEauthorblockA{School of Electrical and\\Computer Engineering\\
University of Texas - San Antonio\\
San Antonio, Texas 78240\\
Email: kastnerkyle@gmail.com}}
\maketitle
\begin{abstract}
Memory caching and architectures are vital to the 
performance of modern computers. One of the key types of processors for high
performance computing and machine learning are Graphics Processor Units (GPU).
These specialized coprocessors are excellent for high-throughput, parallel
floating point computations, and are often used in high performance computing
(HPC) applications. This paper will form a review of three papers: 
\emph{Unifying Primary Cache, Scratch, and Register Memories in a Throughput
Processor}, \emph{A Detailed GPU Cache Model Based on Reuse Distance Theory}, 
and \emph{3D GPU Architecture using Cache Stacking: Performance, Cost, Power
and Thermal analysis}. These papers review a wide variety of topics related to
GPU memory, and also forshadow one of the key future directions for GPU
architectures - stacked (3D) processors and caches. 
\end{abstract}
\IEEEpeerreviewmaketitle

\section{Introduction}
GPU processing has enabled a wide variety of advances in scientific simulation,
computer graphics, and high perfromance computing, largely driven by the need
for highly parallel floating point mathematics. Thoug GPUs were originally
developed in order to provide dedicated processing for graphics rendering in
computer games, they have since been refocused for more general applications by
one of the leading GPU manufacturers, NVidia. By releasing a dedicated
programming language, called CUDA, which allows programminers to write custom
code for GPUs, NVidia effectively created a new market for software driven 
heavy computational hardware. Converting the underlying architectures from 
"graphics specific" to general computation has not been an easy feat, though 
NVidia's future releases will have unified the Tegra (cellphone, tablet,
and basic laptops), and Kepler (workstations, larger standalone units) 
architectures, which should allow programmers to release the same
code on a multitude of platforms. The following papers detail a few advances
which may further the improvement of these devices in our computers, laptops, 
tablets, and cellphones \cite{gtc}.

\section{Overview of GPU Structure}
\subsection{Compute Structure}
The GPU features a multilevel, hierarchical structure. 
The smallest building block is called a \emph{thread}, several of which 
(typically 32) in turn make up a \emph{warp}. One or more warps can be used to
create a \emph{thread block}. Several thread blocks form a \emph{grid} 
\cite{cuda}. The GPU is able to independently schedule warps, in order
to maximize throughput by pipelining results together as they are fed
through the ALU, load-store, and special function units (SFU), which are
pipelined units which cache multiple outputs \cite{whitepaper}. 
They function in a similar manner as other superscalar architectures 
which have been discussed in class so far, except that double-precision
floating point operations are first-class citizens in the Kepler architecture
architecture \cite{notes}. One of the primary optimizations for GPU processing
is to maximize the number of active threads per warp, since warps are always
fully used - meaning that 33 threads will use two warps, one of which is 
fully utilized (32 threads), while the other has 31 masked (inactive) threads.
By utilizing all threads in a warp, throughput can be increased.

\subsection{Memory Structure}
There are several layers of memory sharing in this hierarchical structure. Much
in the same way that some levels of cache are shared between processors in
modern multi-core CPUs, the GPU also features meltiple levels of shared 
memory. In the GPU, each thread possesses its own private 
local memory. Each thread block possess a shared memory for sharing of data, 
results, and communications. The GPU also possesses a third, global memory 
which is shared between all thread blocks.In addition, the GPU must get it's 
data via transport from the CPU, which means that CPU caches, memory, and hard
disk can also play the role of secondary caching for the GPU, though access
time is greatly increased due to PCIe bus transfer delays.

\section{Cache Optimization for Throughput Processors}
\subsection{Overview}
The first paper \cite{throughput} describes a method for optimizing throughput 
in GPUs, by allowing dynamic reconfiguration of the allocation between
registers, cache, and scratchpad (shared memory) depending on workload type.
Current GPUs have fixed allocation limits for these memories, which the authors
show severely hinders throughput for certain workloads, as well as increasing 
energy costs. Though it may seem that register memory should be very different
than scratchpad and cache, in a GPU the registers are built from SRAM, and are
constructed in a way that is very similar to the aforementioned memories. Using
a two-level warp scheduler and a software controlled register hierarchy, the
authors show a range of performance improvements, with the largest gains coming
from applications with highly unbalanced work-types, which are misaligned with
current fixed-allocation schemes. 

\subsection{Application Characterization}
The authors break GPU applications into 4 primary categories:
\begin{itemize}
    \item Register usage
    \item Shared memory usage
    \item Cacheable memory usage
    \item Balanced memory usage
\end{itemize}
Of these 4 types, balanced memory usage programs fit closest to the current
memory setup for GPUs, while each of the other program types need a greater
amount of memory allocated for their respective memory types 
(register|shared|cacheable) than the others. The authors list a table of 
programs which feature particular uses of GPU memory.
Notable register usage programs include DGEMM (key component of linear 
algebra programs) and PCR (tridiagonal solver). Cache limited programs
include MatrixMul (matrix multiply), VectorAdd (vector addition), and Backprop
(neural network training algorithm). Shared memory decomposition programs 
include lu (LU decomposition), while balanced programs include functions 
related to cryptography (AES, SobolQRNG), linear algebra (SGEMV), and signal
processing (DWT, DCT). It is easy to see why a dynamically reconfigurable
memory architecture could be useful, as these algorithms are often tied 
together in order to form a solution pipeline for a given application.

\subsection{Microarchitecture}
The proposed microarchitecture unifies the 32 independent register banks,
shared memory banks, and cache banks of the baseline architecture (1 each 
per thread in a warp). Rather, there are a total of 32 banks per streaming
multiprocessor (SM) in order to keep bandwidth the same. The key enabling
technology for the unified cache is a software controlled register file
hierarchy. By allocating using an automated algorithm to allocate space, 
the authors are able to more effectively utilize GPU processing. 
The parameters of this algorithm are listed here: 
\begin{itemize}
    \item Register. Compiler calculated the number of registers 
        required per thread to avoid spills.
    \item Shared. Programmer specifies shared memory per thread.
    \item Number of Threads. Number of threads is calculated from
        the number of registers, amount of shread memory, and overall 
        capacity.
    \item Cache. Any remaining memory is used as cache.
\end{itemize}

\subsection{Conclusion}
The authors showed a performance improvement of up to 71\%, as well as energy
reduction up to 33\%, with a worst case running time penalty of ~1\%.
By making processor storage more flexible GPUs should be able to perform a 
wider array of tasks, as well as being more adaptable to a variety of workloads
without programmer optimization and tuning when transferring to newer 
architectures.

\section{A Detailed GPU Cache Model Based on Reuse Distance Theory}
\subsection{Overview}
A more optimal GPU cache model is described in \cite{cache}. Sequential
processors often use reuse distance theory to model cache behavior. The authors
have developed the necessary tool to apply this same theory to GPU behavior,
taking into account the hierarchical nature of GPU caching, non-uniform latency,
and cache associativity. Using tools built in C++ and integrated with the 
Ocelot GPU emulator for this modeling, the authors optimized two models of
cache configuration scoring mean absolute error of 6\% and 8\%.

\subsection{Adjustments for Parallel Execution}
The authors made the following adjustments in order to model the GPU cache.
\begin{itemize}
    \item Adjustment of reuse distance theory for parallel execution.
    \item Memory latency, including a new type of miss called a 
        \emph{latency miss}
    \item Associativity modelling
    \item Modeling the effects of miss status holding registers (MSHR)
    \item Warp divergence, to account for the fact that different warps are 
        scheduled independently
\end{itemize}

\subsection{Summary}
The authors found that their improvements to the GPU cache model improved the
mean absolute modeling error to 6-8\% for the tested computational loads. 
Though this modeling is now outdated, since it was applied to the older Fermi
architecture, much of their work on modelling and caching behavior for
parallel execution in GPUs can also be applied to Kepler, AMD Radeon, 
and ARM Mali (Xbox One) architectures as well. The key results from this paper
are an open souce model for Fermi architectures \cite{gpumodel},
and a relative ordering of importance for GPU modelling. Latency modelling
showed improvement of 12.1\% to 6.4\%, cache associativity 9.6\% to 6.4\%,
and a limited benefit from miss status holding registers (7.1\% to 6.4\%).

\section{3D GPU Architecture using Cache Stacking: Performance, Cost, Power and
Thermal Analysis}
\subsection{Overview}
The use of through silicon vias (TSV) coupled with low-power consumption chips
and components, has led to a rise in 3D structures for computer hardware. By 
utilizing a 3D arrangement, GPUs may be able to avoid one of their gretest 
weaknesses - GPU cache sizes and the penalty of a GPU cache miss. With 
CPU -> GPU transfer times in the hundreds of milliseconds, a cache miss in GPU
can be absolutely devastating to performance, and is one of the primary areas
where GPUs lack behind conventional multicore CPUs. Though 
this paper is now fairly outdated \cite{cachestacking} (2009), and focused more
on graphics pipelines than general computing,
the technology described was also described as an enabling technology for
chip design in \cite{gtc}, and there is no doubt that these techniques will
be crucial to increasing the performance of GPUs (and indeed, CPUs)
in the future. The formulas related to production costs are largely irrelevant,
as these estimates were based on 65nm processes. However, the seciton on
thermal performance is still important - as die sizes shrink, it only becomes
more difficult to provide proper cooling.

\subsection{Power/Thermal Issues}
At the time of publication, the authors were unable to find any suitable power
simulators for GPU architectures. Using a die area model developed for 65nm
processes, coupled with power index assumptions (16000 W/Ghz * m\^2), power
leakage assumptions, and hotspot simulation tools, the authors were able to 
model their proposed 3D GPU design. Overall, MRAM and SRAM designs appear 
largely comparable, with the MRAM designs performing slightly better due to
less leakage power. 2 layer designs performed much, much more consistently
as cache sizes increased from 16kB to 2MB. For very small, mini-L1 style caches,
these results appear to show that a 1 layer cache would be sufficient from a 
power perspective, but for any other use a 2 layer cache is preferable.

\subsection{Conclusion}
The authors investigated many aspects of 3D cache design. Though much of the 
work, including the final simulated temperatures and power consumption are now
outdated due to the changes from 65nm processes, the investigations of layered 
caches and the benefits of MRAM are likely still true. The overall approach of 
using TSVs to create 3D architectures is the current driving force behind 
modern GPUs, and as long as designers can effectively minimize heat while 
while surface area is reduced, these techniques should lead to a massive 
increase in compute density.

\section{Final Thoughts}
The papers reviewed \cite{throughput}, \cite{cache}, \cite{cachestacking} each
take a unique approach to optimizing GPU computing processes. Though the oldest
paper \cite{cachestacking} is beginning to show its age, the concepts behind 
power analysis of 3D architectures are absolutely vital. GPU and CPU production
processes are increasingly moving to 3D techniques using TSVs, and knowing how
much heat must be dispersed is just as critical as in 2D designs. The paper on
using reuse distance theory \cite{cache} developed several important concepts
to move the concept of reuse distance from sequential processors to the 
inherently parallel GPU platform. In addition, they also extended an existing 
open source GPU simulation platform (Ocelot GPU simulator), as well as 
releasing their custom C++ model for public use and scrutiny. The third paper 
reviewed \cite{throughput} provided an interesting method for optimizing 
throughput for variable types of process tasks. With the future of GPU 
moving towards general high performance tasks, job specific cache optimization 
will be critical to further adoption, but may come in conflict with current
architecture assumptions. GPU computing is increasingly important in scientific
work, and these advances will be vital improvements for better understanding
our world.

\begin{thebibliography}{1}
\bibitem{gtc}
Jen-Hsun Huang, \emph{GTC 2014 Keynote}, GPU Technology Conference, San Jose, CA.

\bibitem{whitepaper}
NVidia, \emph{Kepler GK110 Architecture Whitepaper}, retrieved May 2014.

\bibitem{cuda}
NVidia, \emph{CUDA Toolkit Documentation v 6.0}, retrieved May 2014.

\bibitem{throughput}
M. Gebhart, S. Keckler, B. Khailany, R. Krashinsky, W. Dally, 
\emph{Unifying Primary Cache, Scratch, and Register File Memories in a 
Throughput Processor}, MICRO 2012.

\bibitem{cache}
C. Nugteren, G. van der Braak, H. Corporaal, H. Bal, 
\emph{A Detailed GPU cache model Based on Reuse Distance Theory}, 
High Performance Computer Architecture, 2014.

\bibitem{gpumodel}
C. Nugteren, G. van der Braak, H. Corporaal, H. Bal, 
\emph{Source code for 'A Detailed GPU cache model Based on Reuse Distance Theory'}
 https://github.com/CNugteren/gpu-cache-model

\bibitem{cachestacking}
A. Al Maashri, G. Sun, X. Dong, V. Narayanan, Y. Xie,
\emph{3D GPU Architecture using Cache Stacking: Performance, Cost, Power, and
Thermal analysis}, IEEE International Conference on Computer Design ICCD 2009.

\bibitem{notes}
W. Lin, \emph{Course Notes for Computer Architecture, Spring 2014}, UTSA, 2014.
\end{thebibliography}




% that's all folks
\end{document}


