
\documentclass[conference]{IEEEtran}
\hyphenation{op-tical net-works semi-conduc-tor}


\begin{document}
%
% paper title
% can use linebreaks \\ within to get better formatting as desired
\title{Modern Memory Caching (Proposal) for Computer Architecture,
Spring 2014, UTSA}

% author names and affiliations
% use a multiple column layout for up to three different
% affiliations
\author{\IEEEauthorblockN{Kyle Kastner}
\IEEEauthorblockA{School of Electrical and\\Computer Engineering\\
University of Texas - San Antonio\\
San Antonio, Texas 78240\\
Email: kastnerkyle@gmail.com}}
\maketitle
\begin{abstract}
Strategies for memory caching have been crucial to advancement 
in computer architectures over the past decades. The proposed research 
paper will review several state of the art techniques in cache management, 
invalidation, and design, in order to understand modern memory caching 
policies and performance for ARM, x86, and CUDA GPU.
\end{abstract}
\IEEEpeerreviewmaketitle

\section{Approach}
Memory caching is crucial to the performance of modern computers. By reviewing 
and analyzing several state of the art papers on caching policies and 
structures, 
the report will provide a review of these techniques and how they are 
applied in modern computer architectures including x86, ARM, and CUDA GPU.
The team will consist of one member, Kyle Kastner, who will cover the previously
mentioned topics.

\section{Deliverables}
Deliverables for this research topic will be one research paper which covers
modern memory caching policies, physical designs, and algorithms. This paper 
will be an analysis and literature review of current publications in this area,
as well as discussion of current cache architectures for 
x86, ARM, and CUDA GPU.

\begin{thebibliography}{1}
\bibitem{notes}
W. Lin, \emph{Course Notes for Computer Architecture, Spring 2014}, UTSA, San Antonio, 2014.
\end{thebibliography}




% that's all folks
\end{document}


